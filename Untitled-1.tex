
¥\RequirePackage{plautopatch}
\RequirePackage[l2tabu, orthodox]{nag}

\documentclass[platex,dvipdfmx]{jlreq}			% for platex
% \documentclass[uplatex,dvipdfmx]{jlreq}		% for uplatex

\usepackage[dvipdfmx]{graphicx}
\usepackage{bxtexlogo}
\usepackage{braket}
\usepackage{here}
\usepackage{amsmath}
\usepackage{physics}
\newcommand\nn{\nonumber \\}
\newcommand\beq{ \begin{qnarray} }
\newcommand\eeq{ \end{qnarray} }
\title{学士論文\\
スピン角運動量を考慮した電子による古典的輻射場の誘導吸収}
\author{学生番号b203p025t 小松 慎}
\date{\today}
\begin{document}
\maketitle
\clearpage
\tableofcontents
\newpage
\section{研究背景・目的}
近年、電子スピンによる情報の伝達、相対論的重イオン衝突実験におけるハドロンのスピン偏極現象など粒子のスピンダイナミクスが関わる物理が興味をひいている。本研究ではスピン1/2の荷電粒子の古典的輻射場の誘導吸収かていを量子力学的に調べる。誘導吸収とは、粒子に外部から電磁波を照射し吸収させることである。
 本研究では、自由粒子を考え、量子数はスピン角運動量と運動量をとり、電磁波の照射する角度がどのように遷移確率に影響するかを、スピンフリップする場合とスピンフリップしない場合について考えていく。また、量子化軸(Z軸)を電場の振動方向、磁場の振動方向、電磁波の進行方向にとり、それぞれ調べた。本研究では、電磁波と荷電粒子との相互作用にスピン角運動量に依存する相対論的補正(ゼーマン効果、スピン軌道力)を取り入れ、その影響について考察する。
\section{古典的輻射場との相互作用への応用}
時間に依存する摂動論を原子内の電子と古典的輻射場との相互作用に対して行うことにする。古典輻射場とは古典的で、量子化されていない電場や磁場である。電荷eを持つ電子がベクトルポテンシャル$\vb{A}$とスカラーポテンシャル$\phi$の電磁場にいるとするとハミルトニアンは以下のように与えられる。
$H=\frac{\vb{p}^2}{2m}+e\phi-\frac{e}{mc}\vb{A}\cdot\vb{p}$
\end{document}


